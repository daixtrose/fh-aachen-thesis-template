% Some great ressources to understand this configuration:
% - https://www.overleaf.com/
%   - Please note that Overleaf provides a very good documentation 
%     about LaTeX in https://www.overleaf.com/learn  
% - https://ctan.org/pkg/koma-script?lang=en

\PassOptionsToPackage{unicode}{hyperref}
\PassOptionsToPackage{hyphens}{url}
\PassOptionsToPackage{dvipsnames,svgnames*,x11names*,table}{xcolor}

\documentclass[11pt,
    a4paper,
    american,
    numbers=noenddot, % not 1.1. but 1.1 
    oneside,
    bibliography=totoc,
    bibtotocnumbered,
    listof=totoc,
    parskip=half,
]{scrreprt}

% fix some packages like float
\usepackage{scrhack}

\usepackage{geometry}

% This will make LaTeX output nag on a lot of issues that otherwise remain undetected
\usepackage[l2tabu]{nag}

% Unicode
\usepackage[utf8]{inputenc}
\usepackage[T1]{fontenc}

% Fonts
\usepackage{helvet}
\usepackage{lmodern}
\renewcommand{\familydefault}{\sfdefault}

% Color
\usepackage{xcolor}

% Graphics
\usepackage{graphicx}
% Search path for images  
\graphicspath{{images/}}

% Math Stuff
\usepackage{amsfonts}
\usepackage{amsmath}
\usepackage{mathabx}
\usepackage{MnSymbol}


% Avoid LaTeX Error: Too many math alphabets used in version normal.
% See https://tex.stackexchange.com/a/243541/144487
\newcommand\hmmax{0}
\newcommand\bmmax{0}

% Mathtools provides a series of packages designed to enhance the appearance 
% of documents containing a lot of mathematics.
% See https://ctan.org/pkg/mathtools?lang=en
\usepackage{mathtools}

% Cross references 
% Packages must be included in this order, 
% see https://tex.stackexchange.com/a/83051/144487
\usepackage{varioref}
\usepackage[pdfencoding=unicode, psdextra]{hyperref}
\usepackage{cleveref}

% A new reference scheme that is orthogonal to the standard stuff
% A rising star if you want to do fancy stuff
\usepackage{zref}

% Generate blind text
\usepackage{lipsum}

% Line spacing
\usepackage{setspace}

% Tables 
% See all table package options in https://www.overleaf.com/learn/latex/Tables
% and how the package tabularx (https://ctan.org/pkg/tabularx?lang=en) can ease your work 
\usepackage{tabularx}
\usepackage{booktabs} % for \midrule
\usepackage{colortbl} % in conjunction with color this creates colored tables


% Special list and enumeration environments
% https://ctan.org/pkg/paralist?lang=en
% \begin{compactitem}[$\circ$]
%     \item you can save some space and
%     \item specify the symbol.
%     \item Let me add ...
% \end{compactitem}
\usepackage{paralist}

%%%%%%%%%%%%%%%%%%%%%%%%%%%%%%%%%%%%%%%%%%%%%%%%%%%%%%%%%%%%%%%%%%%%%%%%%%%%%%%
% Header and Footer
% See the German tutorial on https://tobiw.de/en/teotm/layout-2 
\usepackage{scrlayer-scrpage}

\KOMAoptions{
    headsepline = true,
    footsepline = true,
    plainfootsepline = true,
}

\automark[chapter]{chapter}

\clearpairofpagestyles

\ohead{\headmark}
\ihead{\author}
\ofoot*{\pagemark}

\setkomafont{pageheadfoot}{\sffamily}
\setkomafont{pagination}{\sffamily}


%%%%%%%%%%%%%%%%%%%%%%%%%%%%%%%%%%%%%%%%%%%%%%%%%%%%%%%%%%%%%%%%%%%%%%%%%%%%%%%
% Pgf/Tikz is a powerful graphics package for drawing diagrams, etc.
% See examples at https://texample.net/tikz/examples/
% See documentation at https://ctan.org/pkg/pgf?lang=en
\usepackage{tikz}
\usetikzlibrary{calc,positioning}

% The pgfplots package, which is based on TikZ, is a powerful visualization tool 
% and ideal for creating scientific/technical graphics. The basic idea is that 
% you provide the input data/formula and pgfplots does the rest.
\usepackage{pgfplots}

% Set compatibility mode, see https://tex.stackexchange.com/a/81912/144487
\pgfplotsset{compat=newest}

\usepackage{adjustbox}

\usepackage{xparse}

\usepackage{picture}

\usepackage{siunitx}
\sisetup{per-mode=symbol}

\usepackage{ifthen}

\usepackage{biblatex}

% required for option [H] in figure
\usepackage{float}

%%%%%%%%%%%%%%%%%%%%%%%%%%%%%%%%%%%%%%%%%%%%%%%%%%%%%%%%%%%%%%%%%%%%%%%%%%%%%%%
% Some definitions used later

\DeclareMathOperator{\diverg}{\mathit{div}}
\DeclareMathOperator{\divergref}{\Ref{\mathit{div}}}
\DeclareMathOperator{\grad}{\mathit{grad}}
\DeclareMathOperator{\gradref}{\Ref{\mathit{grad}}}
\DeclareMathOperator{\trace}{\mathit{tr}}
\newcommand{\traceof}[1]{\trace \left( #1 \right)}
\DeclareMathOperator{\doubleshift}{\mathit{:}}
\DeclareMathOperator{\placeholder}{\cdot}
\DeclareMathOperator{\determinant}{\mathit{det}}

\newcommand{\detof}[1]{\left| #1 \right|}
\newcommand{\signum}[1]{\mathit{sign}\left( #1 \right)}
\newcommand{\tvec}[1]{\ensuremath{\boldsymbol{#1}}}
\newcommand{\pfrac}[2]{\displaystyle{\frac{\partial #1}{\partial #2}}}
\newcommand{\pfractwo}[2]{\displaystyle{\frac{\partial^{2} #1}{\partial #2^{2}}}}
\newcommand{\pfracbig}[2]{\displaystyle{\frac{\partial}{\partial #2}} #1}
\newcommand{\pfracshort}[2]{{#1}_{#2}}


\newcommand{\ten}[1]{ \ifthenelse{ \equal{A}{#1} \or \equal{B}{#1} \or
        \equal{C}{#1} \or \equal{D}{#1} \or \equal{E}{#1} \or \equal{C}{#1} \or
        \equal{D}{#1} \or \equal{E}{#1} \or \equal{F}{#1} \or \equal{G}{#1} \or
        \equal{H}{#1} \or \equal{I}{#1} \or \equal{J}{#1} \or \equal{K}{#1} \or
        \equal{L}{#1} \or \equal{M}{#1} \or \equal{N}{#1} \or \equal{P}{#1} \or
        \equal{Q}{#1} \or \equal{R}{#1} \or \equal{S}{#1} \or \equal{T}{#1} \or
        \equal{U}{#1} \or \equal{V}{#1} \or \equal{W}{#1} \or \equal{X}{#1} \or
        \equal{Y}{#1} \or \equal{Z}{#1} }{\ensuremath{\text{$\boldsymbol{#1}$}}}{
        \ensuremath{\text{\Large $\boldsymbol{#1}$}}}}
\newcommand{\stress}{\ensuremath{\sigma}}
\newcommand{\tenstress}{\ensuremath{\ten{\stress}}}
